\documentclass{beamer}
\usetheme{CambridgeUS}
\usecolortheme{dolphin}

\usepackage{CJKutf8}
\usepackage[greek,english]{babel}
\usepackage[OT2,OT1]{fontenc}
\usepackage[utf8]{inputenc}
\usepackage{geometry}
\usepackage{listings} 
\usepackage{enumerate}
\usepackage{tabu} 
\usepackage{multirow}
\usepackage{multicol}
\usepackage{tipa}
\usepackage{ulem}

\newenvironment{command}{\begin{block}{Command}}{\end{block}}
\newcommand{\samplecolorbox}[1]{\fcolorbox{black}{#1}{\color{#1}{\tiny{\phantom{0000}}}} \small{#1}}
\newcommand{\sampletext}[2]{\alert{\textbackslash #1} - {#2{Sample Text}}}
\newcommand{\sampleaccent}[3]{\alert{\textbackslash #1}\{#3\}\quad #2{#3}}
\newcommand{\samplesymbol}[2]{\alert{\textbackslash #1}\quad #2}

\title{Introduction to \LaTeX}
\author{Liu Yihao}
\date{\today}

\begin{document}

\begin{frame}
	\titlepage	
\end{frame}

\section{Getting Started}
\begin{frame}
	\tableofcontents[currentsection,hideothersubsections]
\end{frame}

\subsection{What is \LaTeX}

\begin{frame}
	\frametitle{What is \LaTeX}
	\begin{block}{From Wikipedia, the free encyclopedia}
		LaTeX (lah-tekh, lah-tek or lay-tek, a shortening of Lamport TeX) is a document preparation system. When writing, the writer uses plain text in markup tagging conventions to define the general structure of a document (such as article, book, and letter), to stylise text throughout a document (such as bold and italic), and to add citations and cross-references. A TeX distribution such as TeX Live or MikTeX is used to produce an output file (such as PDF or DVI) suitable for printing or digital distribution. Within the typesetting system, its name is stylised as \LaTeX.
	\end{block}
\end{frame}

\subsection{Installation of \LaTeX}
\begin{frame}
	\frametitle{Installation of \LaTeX}
	\begin{block}{Windows}
		Download TeXLive on the follwing website\\
		\href{http://mirror.hust.edu.cn/CTAN/systems/texlive/Images/}{\color{blue}\underline{http://mirror.hust.edu.cn/CTAN/systems/texlive/Images/}}
	\end{block}
	\begin{block}{Linux}
		For example, on Ubuntu (or Debian), Enter the command\\
		\alert{sudo apt-get install texlive-full}
	\end{block}
	\begin{block}{MacOS}
		Download MacTeX on the following website\\
		\href{http://tug.org/mactex/mactex-download.html}
		{\color{blue}\underline{http://tug.org/mactex/mactex-download.html}}
	\end{block}
\end{frame}

\subsection{Selection of IDEs}

\begin{frame}
	\frametitle{Selection of IDEs}
	There are various IDEs recommended that support \LaTeX , for example\\
	\begin{block}{Texmaker}
		\href{http://www.xm1math.net/texmaker/}{\color{blue}\underline{http://www.xm1math.net/texmaker/}}
	\end{block}
	\begin{block}{Sublime Text}
		\href{http://www.sublimetext.com/}{\color{blue}\underline{http://www.sublimetext.com/}}
	\end{block}
	\begin{block}{Tex Studio}
		\href{http://www.texstudio.org/}{\color{blue}\underline{http://www.texstudio.org/}}
	\end{block}
	They all have cross-platform support for Windows, Linux and MacOS.
\end{frame}

\subsection{Documentation}

\begin{frame}
	\frametitle{Documentation on your computer}
	If you've installed a full version of TeXLive (as strongly recommended), the \LaTeX\ documentation about all you want to is in front of you.\\
	\ \\
	Open the command line and input the command\\
	\alert{texdoc} \structure{docname}\\
	For example, you can use the following types for the \structure{docname}\\
	\begin{description}
		\item[tex] 		A documentation about \structure{TeX}\\
		\item[article] 	A documentation about documentclass \structure{article}\\
		\item[beamer] 	A documentation about documentclass \structure{beamer}\\
		\item[pgf]		A documentation about \structure{TikZ} and \structure{PGF} (used to draw graphs)\\
	\end{description}
	Just try to \alert{texdoc} about all new things then you will be an expert in \LaTeX.
\end{frame}

\section{The Basic Usages}
\begin{frame}
	\tableofcontents[currentsection,hideothersubsections]
\end{frame}

\subsection{Common syntax}

\begin{frame}
	\frametitle{The common syntax of \LaTeX\ commands}
	\begin{definition}
		\structure{Command} is a word which can be identified by Latex and represents a certian function in output file, or in relation with some specific character or format
	\end{definition}
	All \LaTeX\ commands have the following syntax\\
	\alert{\textbackslash command\_name}\textless \structure{special\_args}\textgreater [\structure{optional\_args}]\{\structure{required\_args}\}
	\begin{description}
		\item[special\_args]	Seldom used in basic usage, for certain special usages in some packages
		\item[optional\_args]	Used to define mode of the command, if not specified, \LaTeX\ will use the default mode
		\item[required\_args]	Must be filled
	\end{description}
	If you want to connect a letter after a command, a space must be appended after the command or \LaTeX\ won't be able to compile it correctly. But two commands can be directly connected since there is a \structure{\textbackslash} before each command.
\end{frame}

\begin{frame}
	\frametitle{The common syntax of \LaTeX\ environments}
	\begin{definition}
		\structure{Environment} is an encapsulated part which has a certain format so that it will not be influenced by outer context
	\end{definition}
	All \LaTeX\ environments have the following syntax\\
	\structure{\textbackslash begin}\{\alert{environment\_name}\}\textless \structure{special\_args}\textgreater [\structure{optional\_args}]\\
	\qquad...\\
	\structure{\textbackslash end}\{\alert{environment\_name}\}\\
	\begin{description}
		\item[special\_args]	Similar to commands
		\item[optional\_args]	Similar to commands
	\end{description}
	It is recommended to have a tab indent in each environment or your tex codes will be difficult to read by others or even \alert{yourself}.
\end{frame}

\begin{frame}
	\frametitle{Environment in enviornment}
	Of course, the environments can be nested.\\
	\begin{example}
		\structure{\textbackslash begin}\{\alert{environment\_name}\}\\
		\qquad ...\\
		\qquad\structure{\textbackslash begin}\{\alert{environment\_name\_2}\}\\
		\qquad\qquad ...\\
		\qquad\structure{\textbackslash end}\{\alert{environment\_name\_2}\}\\
		\qquad ...\\
		\structure{\textbackslash end}\{\alert{environment\_name}\}\\
	\end{example}
\end{frame}


\subsection{Documentclass}

\begin{frame}
	\frametitle{All begins with documentclass}
	\begin{definition}
		In a \LaTeX\ file, the {\color{blue}first} line must be \\
		\alert{\textbackslash documentclass}[\structure{options}]\{\structure{class}\}
	\end{definition}
	For example, you can use the following types for the \structure{class}\\
	\begin{description}
		\item[ariticle]	Write a report or an science article
		\item[book] 	Write a book
		\item[beamer]	Produce a lecture silde like this!
	\end{description}
	Actually some options can be added, such as\\[0.5em]
	\alert{\textbackslash documentclass}[11pt,twoside,a4paper]\{article\}\\[0.5em]
	Some details about the \structure{article} class are on the next page. More features about other classes and options can be found in the \LaTeX\ Document on your own.
\end{frame}

\begin{frame}
	\frametitle{The article class}
	The \structure{article} class the most basic class in \LaTeX, it provides you with some normalized structure and format for report writing. So usually you will use the following command as the first line of your tex document\\[0.5em]
	\alert{\textbackslash documentclass}[\structure{options}]\{article\}\\[0.5em]
	Some of the options values are listed below (the default values are \alert{alerted})
	\begin{itemize}
		\item \alert{10pt}, \structure{11pt}, \structure{12pt} - the font size of the document
		\item \structure{a4paper}, \structure{a5paper}, \alert{letterpaper} - the size of paper
		\item \structure{fleqn} - make the math equations left aligned (default middle aligned)
		\item \structure{leqno} - display the serial numbers of math equations on the left (default on the right)
		\item \structure{titlepage}, \alert{notitlepage} - whether to make the title an entire page
		\item \alert{onecolumn}, \structure{twocolumn} - the number of columns of the document
		\item \structure{twoside}, \alert{oneside} - influence the position of something on the page
	\end{itemize}
\end{frame}


\subsection{Document environment}

\begin{frame}
	\frametitle{The document environment}
	\begin{definition}
		An document starts with the \structure{document} environment. A typical  (simplest) example is presented below.
	\end{definition}
	\begin{example}
		\alert{\textbackslash documentclass}[a4paper]\{article\}\\
		\structure{\textbackslash begin}\{document\}\\
		\qquad...\\
		\qquad Hello World!\\
		\qquad...\\
		\structure{\textbackslash end}\{document\}\\
	\end{example}

	All of your contents should be in the document environment. The document environment \alert{MUST} be \alert{unique} in the whole file.
\end{frame}

\subsection{Packages}

\begin{frame}
	\frametitle{Magic of packages}
	Some environments or commands cannot be used directly. In this case,  \structure{packages} should be included between \structure{documentclass} and \structure{document environment}.
	\begin{command}
		\alert{\textbackslash usepackage}[\structure{optional\_args}]\{\structure{name}\}
	\end{command}
	There are some very useful packages that you can \alert{ALWAYS} include:
	\begin{description}
		\item[amsmath] Define various maths environments
		\item[amssymb] Define various maths symbols
		\item[geometry] Adjust the margin, paper size, and etc.
		\item[enumerate] Generate a list like this!
		\item[graphicx] Insert image of all types
	\end{description}
	The usages of these and more packages will be introduced further.
\end{frame}

\subsection{Title, Author and Date}

\begin{frame}
	\frametitle{Title, Author and Date}
	It's very useful to generate a title on the first page of a document, then these commands can be added between \structure{documentclass} and \structure{document environment}.
	\begin{command}
		\alert{\textbackslash title}\{\structure{the title}\}\\
		\alert{\textbackslash author}\{\structure{the author}\}\\
		\alert{\textbackslash date}\{\structure{the date}\}\\
	\end{command}
	You can simply use \alert{\textbackslash date}\{\alert{\textbackslash today}\} to display today's date.\\[0.5em]
	Then in the \structure{document environment}, use the command \alert{\textbackslash maketitle} to generate a title.
\end{frame}

\subsection{Sections}

\begin{frame}
	\frametitle{Dividing into sections}
	\begin{command}
		\alert{\textbackslash section(*)}\{\structure{name}\}\\
		\alert{\textbackslash subsection(*)}\{\structure{name}\}\\
		\alert{\textbackslash subsubsection(*)}\{\structure{name}\}\\
	\end{command}
	The default style of sections is like\\
	\structure{1 Example Section Name}\\
	\structure{1.2 Example Subsection Name}\\
	\structure{1.2.3 Example Subsubsection Name}\\[0.5em]
	If a star(\alert{*}) is added, the sequence number of the section, subsection or subsubsection won't be displayed.\\
	\alert{Notice:} Sections can be sorted into commands, not environments, so it doesn't have \structure{begin} and \structure{end} clauses. However, the whole contents between two sections is belonged to one section
\end{frame}

\section{Use Text in \LaTeX}
\begin{frame}
	\tableofcontents[currentsection,hideothersubsections]
\end{frame}

\subsection{UTF-8 encoding}

\begin{frame}
	\frametitle{Use UTF-8 encoding in \LaTeX}
	UTF-8 encoding is widely used in modern computer applications, so it's useful to include the \structure{inputenc} package and use UTF-8 encoding.
	\begin{command}
		\alert{\textbackslash usepackage}[utf-8]\{inputenc\}
	\end{command}
	\begin{example}
		café
	\end{example}
	However, different operating systems and compiling engines have different support on UTF-8 encoding, some UTF-8 codes that work on your computer may not work on others, so it is recommended to use commands (will be introduced later) instead of directly copy and paste the UTF-8 codes from the Internet.
\end{frame}

\begin{frame}
	(This part is not important)\\
	If you want to use a language other than English, another package \structure{babel} in needed.
	\begin{command}
		\alert{\textbackslash usepackage}[\structure{languages}]\{babel\}
		\begin{itemize}
			\item \structure{languages} - a list of languages, the last one to be the default language
		\end{itemize}
	\end{command}
	\begin{example}
		\alert{\textbackslash usepackage}[greek,english]\{babel\}\\
		\alert{\textbackslash textgreek}\{abcdefgABCDEFG\}\\
	\end{example}
	Then \LaTeX\ will print \textgreek{abcdefgABCDEFG}\\
	Of course, you can use some commands these greek letters, such as \alert{\textbackslash alpha}, \alert{\textbackslash beta} and etc, which is more convenient when you only need to print few of them, and it doesn't need any package listed above.
\end{frame}

\subsection{Special symbols and accents}

\begin{frame}
	\frametitle{Special symbols}
	Some special symbols can't be directly used since they are reserved by \LaTeX
	\begin{center}
	\begin{tabular}{llllll}
		\samplesymbol{\#}{\#} & \samplesymbol{\$}{\$} & \samplesymbol{\%}{\%} & \samplesymbol{\&}{\&} & \samplesymbol{\~{}}{\~{}} & \samplesymbol{\`{}}{\`{}} \\
		\samplesymbol{\{}{\{} & \samplesymbol{\}}{\}} & \samplesymbol{\_}{\_} &
		\multicolumn{2}{l}{\samplesymbol{textbackslash}{\textbackslash}}
	\end{tabular}
	\end{center}
	Many \LaTeX\ starters are confused with how to correctly print quotes, hyphens and dots.\\
	\`{} prints a left single quote, ' prints a right single quote.\\
	\`{}\`{} prints a left double quote, '' prints a right double quote.\\
	one hyphen (-) print like - \\
	two hyphens ({-}{-}) print like -- \\
	three hyphens ({-}{-}{-}) print like ---\\
	\alert{\textbackslash dots} prints the dots with a correct format (\dots) instead of directly use three dots (...)
\end{frame}

\begin{frame}
	\frametitle{Accent on letters}
	Sometimes you may need an accent form of a letter, here is an example of letter \structure{o}
	\begin{center}
	\begin{tabular}{lllll}
		\sampleaccent{\`{}}{\`}{o} & \sampleaccent{'}{\'}{o} & \sampleaccent{\^{}}{\^}{o} & \sampleaccent{"}{\"}{o} & \sampleaccent{\~{}}{\~}{o} \\
		\sampleaccent{=}{\=}{o} & \sampleaccent{.}{\.}{o} & \sampleaccent{u}{\u}{o} & \sampleaccent{v}{\v}{o} & \sampleaccent{H}{\H}{o}\\
		\sampleaccent{t}{\t}{oo} & \sampleaccent{r}{\r}{o} & \sampleaccent{c}{\c}{o} & \sampleaccent{d}{\d}{o} & \sampleaccent{b}{\b}{o}
	\end{tabular}
	\end{center}
	\begin{block}{Something interesting}
		You may be curious about how to print words like \LaTeX, actually it's defined as a command.
		\begin{itemize}
			\item \alert{\textbackslash TeX} - \TeX
			\item \alert{\textbackslash LaTeX} - \LaTeX
			\item \alert{\textbackslash LaTeXe} - \LaTeXe
		\end{itemize}
	\end{block}
\end{frame}

\subsection{Spaces, lines and pages}

\begin{frame}
	\frametitle{Spaces may be confusing}
	There are defined command of spaces in different width and usages.
	\begin{itemize}
		\item \colorbox{yellow}{\ } - the basic space in \LaTeX\ (printed in yellow since it's transparent). Note that any number of spaces or tabs is equal to one space, and the space after a command is ignored. If you want to add an extra space, use \alert{\textbackslash}\colorbox{yellow}{\ } which makes a 1/3\,em space (1 em is approximately the width of an \structure{M} in the current font)
		\item \~{} - If two words can't be separated on two lines, you can tell \LaTeX\ about it using a tie (\~{}), such as Prof.\~{}Hamade (Prof.~Hamade).
		\item  \alert{\textbackslash ,} - makes a 1/6\,em space, commonly used before units (notice the space before em on this page)
		\item  \alert{\textbackslash ;} - makes a 2/7\,em space
		\item  \alert{\textbackslash quad} - makes a 1\,em space
		\item  \alert{\textbackslash qquad} - makes a 2\,em space
		\item  \alert{\textbackslash phantom}\{\structure{text}\} - makes actually the space of \structure{text}, but \structure{text} will be invisible.
	\end{itemize}
\end{frame}

\begin{frame}
	\frametitle{Separate contents into lines and pages}
	Here are some basic commands about lines and pages in \LaTeX, you will use them everywhere.
	\begin{itemize}
		\item \alert{\textbackslash newline} - begin a new line
		\item \alert{\textbackslash\textbackslash} - begin a new line
		\item \alert{\textbackslash\textbackslash[offset]} - begin a new line with an offset
		\item \alert{\textbackslash linebreak} - begin a new line with the words discrete
		\item \alert{\textbackslash newpage} - begin a new page
		\item \alert{\%} - begin a line comment
	\end{itemize}
\end{frame}

\subsection{Fonts}

\begin{frame}
	\frametitle{Basic commands about fonts}
	First, lets start with some commands that transform font types
	\begin{itemize}
		\item \sampletext{bf}{\bf}
		\item \sampletext{it}{\it}
		\item \sampletext{rm}{\rm}
		\item \sampletext{sc}{\sc}
		\item \sampletext{sf}{\sf}
		\item \sampletext{sl}{\sl}
		\item \sampletext{tt}{\tt}
	\end{itemize}
	Note that the commands that transform font types influence the text in the whole scope (\structure{\{...\}}) until another font type is specified. For example, how to use the first command \alert{\textbackslash bf} is shown below\\[0.5em]
	\{\alert{\textbackslash bf} Sample Text\}
\end{frame}

\begin{frame}
	Sometimes we don't want to transform the font types, instead, we can only change the font type of some specified text, then the following commands are used (you can similarly use all font types on the previous page)
	\begin{itemize}
		\item \sampletext{textbf}{\textbf}
		\item \sampletext{textit}{\textit}
		\item \sampletext{textsc}{\textsc}
	\end{itemize}
	However, in a math environment (will be introduced later), some other commands should be used
	\begin{itemize}
		\item \alert{\textbackslash mathbf} - $\mathbf{Sample\ Text}$
		\item \alert{\textbackslash mathit} - $\mathit{Sample\ Text}$
		\item \alert{\textbackslash mathsf} - $\mathsf{Sample\ Text}$
	\end{itemize}
	Note that the math environment doesn't include all of the font types on the previous page. More information about font types can be found \href{http://www.cnblogs.com/make217/p/6123532.html}{\color{blue}\underline{here}}.
\end{frame}

\begin{frame}
	Font size can also be easily modified
	\begin{itemize}
		\item \sampletext{tiny}{\tiny}
		\item \sampletext{scriptsize}{\scriptsize}
		\item \sampletext{footnotesize}{\footnotesize}
		\item \sampletext{small}{\small}
		\item \sampletext{normalsize}{\normalsize}
		\item \sampletext{large}{\large}
		\item \sampletext{Large}{\Large}
		\item \sampletext{LARGE}{\LARGE}
		\item \sampletext{huge}{\huge}
		\item \sampletext{Huge}{\Huge}
	\end{itemize}
\end{frame}

\begin{frame}
	\frametitle{Build a colorful document}
	Changing the color is similar to changing font types.\\[0.5em]
	If you want to transform to a color (like \alert{\textbackslash bf}), you can use \alert{\textbackslash color}\{\structure{name}\}\\
	Similarly, you can use \alert{\textbackslash textcolor}\{\structure{name}\} like \alert{\textbackslash textbf}\\
	The background color of the whole page can be set using \alert{\textbackslash pagecolor}\{\structure{name}\}\\[0.5em]
	There are some defined color \structure{name} in the \structure{xcolor} package.\\[0.5em]
	\begin{tabular}{lllll}
	\samplecolorbox{black}&\samplecolorbox{gray}&\samplecolorbox{olive}&\samplecolorbox{teal}&\samplecolorbox{blue}\\
	\samplecolorbox{green}&\samplecolorbox{orange}&\samplecolorbox{violet}&\samplecolorbox{brown}&\samplecolorbox{lightgray}\\
	\samplecolorbox{pink}&\samplecolorbox{white}&\samplecolorbox{cyan}&\samplecolorbox{lime}&\samplecolorbox{purple}\\
	\samplecolorbox{yellow}&\samplecolorbox{darkgray}&\samplecolorbox{magenta}&\samplecolorbox{red}\\
	\end{tabular}		
	\\[0.5em]
	You can find more information in the documentation of \structure{xcolor} (\alert{texdoc} \structure{xcolor})
\end{frame}

\subsection{Enumerate}
\begin{frame}
	\frametitle{Enumerate and Item}
	When you need to enumerate some items as a list, you may use the \structure{enumerate} package.
	\begin{command}
		\alert{\textbackslash usepackage}\{enumerate\}\\
		\structure{\textbackslash begin}\{enumerate\}[\structure{style}]\\
		\qquad\alert{\textbackslash item} ...\\
		\qquad\alert{\textbackslash item} ...\\
		\qquad\alert{\textbackslash item} ...\\
		\structure{\textbackslash end}\{enumerate\}
	\end{command}
	This will generate a normal list with the serial numbers in the specified \structure{style}, which could be the following (as example)
	\begin{itemize}
		\item \alert{1} - 1, 2, 3, 4, ...
		\item \alert{(i)} - (i), (ii), (iii), (iv), ...
		\item \alert{[1.]} - [1.], [2.], [3.], [4.], ...
	\end{itemize}
\end{frame}

\begin{frame}
	If you want to generate an unordered list, use \structure{itemize} instead of \structure{enumerate}.
	\begin{command}
		\alert{\textbackslash usepackage}\{enumerate\}\\
		\structure{\textbackslash begin}\{itemize\}\\
		\qquad\alert{\textbackslash item}[\structure{style}] ...\\
		\qquad\alert{\textbackslash item}[\structure{style}] ...\\
		\qquad\alert{\textbackslash item}[\structure{style}] ...\\
		\structure{\textbackslash end}\{itemize\}
	\end{command}
	In this case, the position of \structure{style} is different from that in \structure{enumerate}, and the symbol displayed in the beginning of each item will be exactly same as the \structure{style}. If \structure{style} is not added, a default style will be used.
\end{frame}

\subsection{Lines on text}

\begin{frame}
	\frametitle{Ulem package}
	If you want to add some lines on the text, use the \structure{ulem} package.
	\begin{command}
		\alert{\textbackslash usepackage}\{ulem\}\\
		\alert{\textbackslash uline}\{Sample Text\}
	\end{command}
	There are different kinds of lines supported:
	\begin{itemize}
		\item \sampletext{uline}{\uline}
		\item \sampletext{uuline}{\uuline}
		\item \sampletext{uwave}{\uwave}
		\item \sampletext{sout}{\sout}
		\item \sampletext{xout}{\xout}
		\item \sampletext{dashuline}{\dashuline}
		\item \sampletext{dotuline}{\dotuline}
	\end{itemize}
\end{frame}

\section{Use Maths in \LaTeX}
\begin{frame}
	\tableofcontents[currentsection,hideothersubsections]
\end{frame}

\subsection{Basic equations}

\begin{frame}
	\frametitle{The equation environment}
	An \structure{equation} enviornment contains a set of maths equations
	\begin{command}
		\alert{\textbackslash begin}\{euqation(\structure{*})\}\\
		\qquad ...\\
		\alert{\textbackslash end}\{euqation(\structure{*})\}\\
	\end{command}
	\begin{example}
		\begin{equation}
		curl\ F=\left(\frac{\partial F_z}{\partial y}-\frac{\partial F_y}{\partial z}\right)\hat{n_x}+\left(\frac{\partial F_x}{\partial z}-\frac{\partial F_z}{\partial x}\right)\hat{n_y}+\left(\frac{\partial F_y}{\partial x}-\frac{\partial F_x}{\partial y}\right)\hat{n_z}
		\end{equation}
	\end{example}
	If a star(\structure{*}) is added, the sequence number of the equation won't be displayed. Note that the environment name in the \alert{\textbackslash begin} and \alert{\textbackslash end} statements must be the same(both or neither have a \structure{*} here).
\end{frame}

\begin{frame}
	The \LaTeX\ script of the equation above is quite long, but not so difficult as you think so, while how I display the script to you is far more confusing, and you may check it in the tex file of the lecture sildes
	\begin{align*}
	\color{blue} curl\backslash\ F
	\text{=}\backslash left & \color{blue} (\backslash frac\{\backslash partial\ F\_z\}\{\backslash partial\ y\}\\
	& \color{blue} \text{-}\backslash frac\{\backslash partial\ F\_y\}\{\backslash partial\ z\}\backslash right)\backslash hat\{n\_x\}\\
	\color{blue} \text{+}\backslash left & \color{blue} (\backslash frac\{\backslash partial\ F\_x\}\{\backslash partial\ z\}\\
	& \color{blue} \text{-}\backslash frac\{\backslash partial\ F\_z\}\{\backslash partial\ x\}\backslash right)\backslash hat\{n\_y\}\\
	\color{blue} \text{+}\backslash left & \color{blue} (\backslash frac\{\backslash partial\ F\_y\}\{\backslash partial\ x\}\\
	& \color{blue} \text{-}\backslash frac\{\backslash partial\ F\_x\}\{\backslash partial\ y\}\backslash right)\backslash hat\{n\_z\}
	\end{align*}
	In the script, only a space after \structure{\textbackslash} will be printed as a space, \structure{\textbackslash partial} prints the symbol \structure{$\partial$}, \structure{\textbackslash frac\{...\}\{...\}} makes a \structure{fraction}, \structure{\textbackslash left(} and \structure{\textbackslash right)} makes \structure{brackets} (of course they can be nested and must be in couple, but you can use two kinds of brackets on the both side, i.e., \structure{\textbackslash left[} and \structure{\textbackslash right\textbackslash rbrace}, in which you must use \structure{\textbackslash rbrace} or \structure{\textbackslash \}} to print a right brace \structure{$\rbrace$} \\
\end{frame}

\begin{frame}
	How about equations with multiple lines?\\
	The \structure{aligned} enviornment can be used.
	\begin{example}
		\begin{equation}
		\left\lbrace\begin{aligned}
			x+y&=1\\x-y&=1
		\end{aligned}\right.\Longrightarrow
		\left\lbrace\begin{aligned}
			x&=1\\y&=0
		\end{aligned}\right.
		\end{equation}
	\end{example}
	\begin{align*}
		&\color{blue} \backslash left \backslash lbrace \backslash begin\{aligned\}\\
		&\color{blue} \quad\quad\text{x+y\&=1} \backslash \backslash \text{x-y\&=1}\\
		&\color{blue} \backslash end\{aligned\}\backslash right.\backslash Longrightarrow\\
		&\color{blue} \backslash left \backslash lbrace \backslash begin\{aligned\}\\
		&\color{blue} \quad\quad\text{x\&=1} \backslash \backslash \text{y\&=0}\\
		&\color{blue} \backslash end\{aligned\}\backslash right.
	\end{align*}
	We can use a dot(\structure{.}) when we want to insert nothing in one of the brackets.
\end{frame}

\subsection{Aligns}

\begin{frame}
	\frametitle{The align/aligned enviornment}
	\begin{definition}
		An \structure{align} enviornment is used outside a maths enviornment like \structure{equation}
		{\color{red}\textbackslash begin\{align(*)\}}\\
		\quad ...\\
		{\color{red}\textbackslash end\{align(*)\}}\\
	\end{definition}
	\begin{definition}
		An \structure{aligned} enviornment is used inside a maths enviornment like \structure{equation}
		{\color{red}\textbackslash begin\{euqation(*)\}}\\
			\quad{\color{red}\textbackslash begin\{aligned\}}\\
				\quad\quad ...\\
			\quad{\color{red}\textbackslash end\{aligned\}}\\
		{\color{red}\textbackslash end\{euqation(*)\}}\\
	\end{definition}
	Other properties of them are very similar.
\end{frame}

\begin{frame}
	The \structure{align/aligned} enviornment is an basic align and multiline enviornment.\\
	\begin{example}
		\begin{flalign}
			a+b&\Leftrightarrow b+a\\
			(a+b)+c&\Leftrightarrow a+(b+c)
		\end{flalign}
	\end{example}
	\begin{align*}
		&\color{blue} \backslash begin\{align\}\\
		&\color{blue} \quad\quad a\text{+}b\ \&\ \backslash Leftrightarrow\ b\text{+}a\ \backslash\backslash \\
		&\color{blue} \quad\quad (a\text{+}b)\text{+}c\ \&\ \backslash Leftrightarrow\ a\text{+}(b\text{+}c)\\
		&\color{blue} \backslash end\{align\}
	\end{align*}
	In order to make a new line, you can easily use \structure{\textbackslash\textbackslash} where you'd like (but not in certain maths enviornments such as \structure{equation}). \structure{\&} is used to align the equations, you can use multiple \structure{\&}s and the \structure{\&}s on every line will be aligned respectively.
\end{frame}


\begin{frame}
	\frametitle{Something more about equation environment}
    What if the space between equation and the main body paragraph is considered larger than expectation? Is there any way to modify the line spaceing?
\\In default style of equation is like
    \begin{example}
    your body paragraph is supposed to be typed here
        \begin{equation}
         a \times b =c
        \end{equation}
    your body paragraph is supposed to be typed here
	\end{example}
\end{frame}
\begin{frame}
But if we add 
{\color{red} $\backslash$setlength$\backslash$abovedisplayskip or belowdisplayskip( pt)} before the equation environment, we have
    \begin{example}
    your body paragraph is supposed to be typed here
		{\setlength\abovedisplayskip{0pt}
        \setlength\belowdisplayskip{0pt}
        \begin{equation}
         a \times b =c
        \end{equation}}
    your body paragraph is supposed to be typed here
	\end{example}
    \structure{\{$\backslash$setlength$\backslash$abovedisplayskip\{0pt\}
        \\$\backslash$setlength$\backslash$belowdisplayskip\{0pt\}
       \\ $\backslash$begin\{equation\}
         \\a $\backslash$times b =c
        \\$\backslash$end\{equation\}\}}
    \\The margin between the body paragraphs and the equation will be lessened 
    as is in the example.
\end{frame}
\begin{frame}
	\frametitle{Typing subscript or superscript out side an equation environment}
    Sometimes you may encounter such circumstance as typing subscript or superscript in your paragraph without entering an equation environment.
    \begin{example}
       The concentration of [H$_3$O$^+$]
    \end{example}
    [H$_3$O$^+$] is typed in the following way,
    \\\structure{ [H\textdollar \textunderscore\{3\} \textdollar O \textdollar \textasciicircum \{+\} \textdollar] }
    \\ in which we type every subscript with 
    \\{\color{red} \textdollar \textunderscore\{anyting you wanna enter as subscript\}\textdollar}
    \\every subscript with 
    \\{\color{red} \textdollar \textasciicircum\{anyting you wanna enter as subscript\}\textdollar}
    \\ if we only want to type one character as the subscript or superscript \{\} can be omiited.
\end{frame}
\begin{frame}
	\frametitle{Draw a table}
    \begin{example}
        \begin{tabular}{|l|c|r|}
        \hline
        Title 1 & Title 2 & Title 3 \\
        \hline
        1 & 2 &3 \\
        \hline
        \end{tabular}
    \end{example}
    The example above goes like this:\\
        $\backslash$begin\{tabular\}\{\textvertline l\textvertline c\textvertline r\textvertline \} \%l represents aligning left; 
        \\ \qquad \qquad \qquad \qquad \qquad \quad c represents centering; 
        \\ \qquad \qquad \qquad \qquad \qquad \quad r represents aligning right
        \\ \qquad \qquad \qquad \qquad \qquad \quad \textvertline \, means the vertical frame of a column\\
        $\backslash$hline \quad \% hline means to draw a horizontal line for all columns\\
        Title 1\&Title 2\&Title 3  \%\& is used to divide contents of different columns\\
        $\backslash$hline\\
        1 \& 2 \&3 \\
        $\backslash$hline\\
        $\backslash$end\{tabular\}
\end{frame}
\begin{frame}
   \frametitle{Something more about tabular}
   $\backslash$multirow \\
   $\backslash$multicolumn \\
   $\backslash$cline{} \\
\end{frame}
\begin{frame}
	\frametitle{Table environment}
    \begin{definition}
    A \structure{table} enviornment is used to arrange the place of a tabular
		{\color{red}\textbackslash begin\{table(*)\}[htbp]}\\
		\quad ...\\
		{\color{red}\textbackslash end\{table(*)\}}\\
	\end{definition}
    \text{[h]} means inserting the tabular to the current place.
    \\\text{[t]} means inserting the tabular to the top of the page.
    \\\text{[b]} means inserting the tabular to the bottom of the page.
    \\\text{[p]} means inserting the tabular to another new page, which is common in dealing with big table.
\end{frame}
\begin{frame}
	\frametitle{Insert a graph}
    
\end{frame}



\begin{frame}

\end{frame}

\section{Use Graphics in \LaTeX}
\begin{frame}
	\tableofcontents[currentsection,hideothersubsections]
\end{frame}

\subsection{Include graphs}

\begin{frame}

\end{frame}

\subsection{Draw graphs}

\begin{frame}

\end{frame}

\section{Some Advantage Usages}
\begin{frame}
	\tableofcontents[currentsection,hideothersubsections]
\end{frame}

\subsection{New and renew}

\begin{frame}

\end{frame}

\section{References}
\begin{frame}
	\tableofcontents[currentsection,hideothersubsections]
\end{frame}

\subsection{Symbol table}

\begin{frame}

\end{frame}

\subsection{Package List}

\begin{frame}

\end{frame}

\subsection{Contributors}

\begin{frame}
	\frametitle{Reference resources}
	\begin{CJK*}{UTF8}{gbsn}
	\begin{itemize}
		\item \LaTeX\ 入门, Haiyang Liu, Publishing House of Electronics Industry, 2013.6, ISBN 978-7-121-20208-7
		\item Introduction to \LaTeX, David Reid, \href{https://wenku.baidu.com/view/f08fbdf24693daef5ef73d23.html}{\color{blue}\uline{https://wenku.baidu.com/view/f08fbdf24693daef5ef73d23.html}}
	\end{itemize}
	\end{CJK*}
\end{frame}

\begin{frame}
	\frametitle{Contributors}
	This \LaTeX\ beamer slide is contributed to
	\begin{itemize}
	\item Liu Yihao (https://github.com/tc-imba)\\
	\item Zhou Yanjun (https://github.com/AuroraZK)\\
	\item Zhang Yifei (https://github.com/zhangyifei-chelsea)
	\end{itemize}	
	For \LaTeX\ lectures of the JI Technology Department.\\
	For all students in JI as a reference in report/homework writing.\\[0.5em]
	
	This is a long-term maintained project on \href{https://github.com/SJTU-UMJI-Tech/LaTeX}{\color{blue}\underline{GitHub}}, if you have any suggestions, make an issue on it, PRs are welcomed as well.
	
\end{frame}


\end{document}
