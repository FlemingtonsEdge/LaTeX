\section{The Basic Usages}
\begin{frame}
	\tableofcontents[currentsection,hideothersubsections]
\end{frame}

\subsection{Common syntax}

\begin{frame}
	\frametitle{The common syntax of \LaTeX\ commands}
	\begin{definition}
		\structure{Command} is a word which can be identified by Latex and represents a certain function in output file, or in relation with some specific character or format
	\end{definition}
	All \LaTeX\ commands have the following syntax\\
	\samplecommand{command\_name}\textless \structure{special\_args}\textgreater [\structure{optional\_args}]\{\structure{required\_args}\}
	\begin{description}
		\item[special\_args]	Seldom used in basic usage, for certain special usages in some packages
		\item[optional\_args]	Used to define mode of the command, if not specified, \LaTeX\ will use the default mode
		\item[required\_args]	Must be filled
	\end{description}
	If you want to connect a letter after a command, a space must be appended after the command or \LaTeX\ won't be able to compile it correctly. But two commands can be directly connected since there is a \structure{\textbackslash} before each command.
\end{frame}

\begin{frame}
	\frametitle{The common syntax of \LaTeX\ environments}
	\begin{definition}
		\structure{Environment} is an encapsulated part which has a certain format so that it will not be influenced by outer context
	\end{definition}
	All \LaTeX\ environments have the following syntax\\
	\samplebegin{\alert{environment\_name}}\textless \structure{special\_args}\textgreater [\structure{optional\_args}]\\
	\qquad...\\
	\sampleend{\alert{environment\_name}}\\
	\begin{description}
		\item[special\_args]	Similar to commands
		\item[optional\_args]	Similar to commands
	\end{description}
	It is recommended to have a tab indent in each environment or your tex codes will be difficult to read by others or even \alert{yourself}.
\end{frame}

\begin{frame}
	\frametitle{Environment in enviornment}
	Of course, the environments can be nested.\\
	\begin{example}
		\samplebegin{\alert{environment\_name}}\\
		\qquad ...\\
		\qquad\samplebegin{\alert{environment\_name\_2}}\\
		\qquad\qquad ...\\
		\qquad\sampleend{\alert{environment\_name\_2}}\\
		\qquad ...\\
		\sampleend{\alert{environment\_name}}\\
	\end{example}
\end{frame}


\subsection{Documentclass}

\begin{frame}
	\frametitle{All begins with documentclass}
	\begin{definition}
		In a \LaTeX\ file, the {\color{blue}first} line must be \\
		\samplecommand{documentclass}[\structure{options}]\{\structure{class}\}
	\end{definition}
	For example, you can use the following types for the \structure{class}\\
	\begin{description}
		\item[ariticle]	Write a report or an science article
		\item[book] 	Write a book
		\item[beamer]	Produce a lecture silde like this!
	\end{description}
	Actually some options can be added, such as\\[0.5em]
	\samplecommand{documentclass}[11pt,twoside,a4paper]\{article\}\\[0.5em]
	Some details about the \structure{article} class are on the next page. More features about other classes and options can be found in the \LaTeX\ Document on your own.
\end{frame}

\begin{frame}
	\frametitle{The article class}
	The \structure{article} class the most basic class in \LaTeX, it provides you with some normalized structure and format for report writing. So usually you will use the following command as the first line of your tex document\\[0.5em]
	\samplecommand{documentclass}[\structure{options}]\{article\}\\[0.5em]
	Some of the options values are listed below (the default values are \alert{alerted})
	\begin{itemize}
		\item \alert{10pt}, \structure{11pt}, \structure{12pt} - the font size of the document
		\item \structure{a4paper}, \structure{a5paper}, \alert{letterpaper} - the size of paper
		\item \structure{fleqn} - make the math equations left aligned (default middle aligned)
		\item \structure{leqno} - display the serial numbers of math equations on the left (default on the right)
		\item \structure{titlepage}, \alert{notitlepage} - whether to make the title an entire page
		\item \alert{onecolumn}, \structure{twocolumn} - the number of columns of the document
		\item \structure{twoside}, \alert{oneside} - influence the position of something on the page
	\end{itemize}
\end{frame}


\subsection{Document environment}

\begin{frame}
	\frametitle{The document environment}
	\begin{definition}
		An document starts with the \structure{document} environment. A typical  (simplest) example is presented below.
	\end{definition}
	\begin{example}
		\samplecommand{documentclass}[a4paper]\{article\}\\
		\samplebegin{document}\\
		\qquad...\\
		\qquad Hello World!\\
		\qquad...\\
		\sampleend{document}\\
	\end{example}

	All of your contents should be in the document environment. The document environment \alert{MUST} be \alert{unique} in the whole file.
\end{frame}

\subsection{Packages}

\begin{frame}
	\frametitle{Magic of packages}
	Some environments or commands cannot be used directly. In this case,  \structure{packages} should be included between \structure{documentclass} and \structure{document environment}.
	\begin{command}
		\samplecommand{usepackage}[\structure{optional\_args}]\{\structure{name}\}
	\end{command}
	There are some very useful packages that you can \alert{ALWAYS} include:
	\begin{description}
		\item[amsmath] Define various maths environments
		\item[amssymb] Define various maths symbols
		\item[geometry] Adjust the margin, paper size, and etc.
		\item[enumerate] Generate a list like this!
		\item[graphicx] Insert image of all types
	\end{description}
	The usages of these and more packages will be introduced further.
\end{frame}

\subsection{Title, Author and Date}

\begin{frame}
	\frametitle{Title, Author and Date}
	It's very useful to generate a title on the first page of a document, then these commands can be added between \structure{documentclass} and \structure{document environment}.
	\begin{command}
		\samplecommand{title}\{\structure{the title}\}\\
		\samplecommand{author}\{\structure{the author}\}\\
		\samplecommand{date}\{\structure{the date}\}\\
	\end{command}
	You can simply use \samplecommand{date}\{\samplecommand{today}\} to display today's date.\\[0.5em]
	Then in the \structure{document environment}, use the command \samplecommand{maketitle} to generate a title.
\end{frame}

\subsection{Sections}

\begin{frame}
	\frametitle{Dividing into sections}
	\begin{command}
		\samplecommand{section(*)}\{\structure{name}\}\\
		\samplecommand{subsection(*)}\{\structure{name}\}\\
		\samplecommand{subsubsection(*)}\{\structure{name}\}\\
	\end{command}
	The default style of sections is like\\
	\structure{1 Example Section Name}\\
	\structure{1.2 Example Subsection Name}\\
	\structure{1.2.3 Example Subsubsection Name}\\[0.5em]
	If a star(\alert{*}) is added, the sequence number of the section, subsection or subsubsection won't be displayed.\\
	\alert{Notice:} Sections can be sorted into commands, not environments, so it doesn't have \structure{begin} and \structure{end} clauses. However, the whole contents between two sections is belonged to one section
\end{frame}

\subsection{Geometry}

\begin{frame}
	\frametitle{Geometry package}
	The settings of the layout of the pages is in \structure{geometry} package.
	\begin{command}
		\samplecommand{usepackage}\{geometry\}\\
		\samplecommand{geometry}\{\structure{options}\}
	\end{command}
	Some of the \structure{options} are listed below:
	\begin{itemize}
		\item \structure{paper} - same as the paper settings in documentclass
		\item \structure{layout} - use another type of paper's layout
		\item \structure{left/right} - the blank length on the left/right
		\item \structure{top/bottom} - the blank length on the top/bottom
	\end{itemize}
	\begin{example}
		\samplecommand{geometry}\{paper=a4paper,layout=a5paper\}
		\samplecommand{geometry}\{left=2.5cm,right=2.5cm,top=2.5cm,bottom=2.5cm\}
	\end{example}
\end{frame}