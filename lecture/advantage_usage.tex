\section{Some Advantage Usages}
\begin{frame}
	\tableofcontents[currentsection,hideothersubsections]
\end{frame}

\subsection{New and renew}

\begin{frame}

\end{frame}
6
\subsection{Input Chinese}

\begin{frame}
	\songti
	\frametitle{输入中文}
	\qquad 虽然密院并不需要输入中文,但为了证明我会在\LaTeX 中输入中文,很有必要在最后一页介绍一下。\\
	\qquad 这里我使用一种较为简便的方法:ctex+XeLaTeX,首先,使用 \structure{ctex} 宏包(\samplecommand{usepackage}\{ctex\}),然后切换到 \structure{XeLaTeX} 编译环境。在 \structure{ctex} 中,已经定义了几个常用的字体,如宋体(\samplecommand{songti}),仿宋(\samplecommand{fangsong}),楷书(\samplecommand{kaishu})等。
	\begin{exampleblock}{例}
		\samplecommand{usepackage}[scheme=plain]\{ctex\}\\
		\samplecommand{songti}\\
		这个功能你们应该是用不到的。
	\end{exampleblock}
	在选项中加入\structure{[scheme=plain]}后可以防止文档的标题,章节等英文字段被汉化。
\end{frame}
